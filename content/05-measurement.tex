% Options for packages loaded elsewhere
% Options for packages loaded elsewhere
\PassOptionsToPackage{unicode}{hyperref}
\PassOptionsToPackage{hyphens}{url}
%
\documentclass[
  ignorenonframetext,
  aspectratio=169,
]{beamer}
\newif\ifbibliography
\usepackage{pgfpages}
\setbeamertemplate{caption}[numbered]
\setbeamertemplate{caption label separator}{: }
\setbeamercolor{caption name}{fg=normal text.fg}
\beamertemplatenavigationsymbolshorizontal
% remove section numbering
\setbeamertemplate{part page}{
  \centering
  \begin{beamercolorbox}[sep=16pt,center]{part title}
    \usebeamerfont{part title}\insertpart\par
  \end{beamercolorbox}
}
\setbeamertemplate{section page}{
  \centering
  \begin{beamercolorbox}[sep=12pt,center]{section title}
    \usebeamerfont{section title}\insertsection\par
  \end{beamercolorbox}
}
\setbeamertemplate{subsection page}{
  \centering
  \begin{beamercolorbox}[sep=8pt,center]{subsection title}
    \usebeamerfont{subsection title}\insertsubsection\par
  \end{beamercolorbox}
}
% Prevent slide breaks in the middle of a paragraph
\widowpenalties 1 10000
\raggedbottom
\AtBeginPart{
  \frame{\partpage}
}
\AtBeginSection{
  \ifbibliography
  \else
    \frame{\sectionpage}
  \fi
}
\AtBeginSubsection{
  \frame{\subsectionpage}
}
\usepackage{iftex}
\ifPDFTeX
  \usepackage[T1]{fontenc}
  \usepackage[utf8]{inputenc}
  \usepackage{textcomp} % provide euro and other symbols
\else % if luatex or xetex
  \usepackage{unicode-math} % this also loads fontspec
  \defaultfontfeatures{Scale=MatchLowercase}
  \defaultfontfeatures[\rmfamily]{Ligatures=TeX,Scale=1}
\fi
\usepackage{lmodern}

\usetheme[]{Madrid}
\ifPDFTeX\else
  % xetex/luatex font selection
\fi
% Use upquote if available, for straight quotes in verbatim environments
\IfFileExists{upquote.sty}{\usepackage{upquote}}{}
\IfFileExists{microtype.sty}{% use microtype if available
  \usepackage[]{microtype}
  \UseMicrotypeSet[protrusion]{basicmath} % disable protrusion for tt fonts
}{}
\makeatletter
\@ifundefined{KOMAClassName}{% if non-KOMA class
  \IfFileExists{parskip.sty}{%
    \usepackage{parskip}
  }{% else
    \setlength{\parindent}{0pt}
    \setlength{\parskip}{6pt plus 2pt minus 1pt}}
}{% if KOMA class
  \KOMAoptions{parskip=half}}
\makeatother


\usepackage{longtable,booktabs,array}
\usepackage{calc} % for calculating minipage widths
\usepackage{caption}
% Make caption package work with longtable
\makeatletter
\def\fnum@table{\tablename~\thetable}
\makeatother
\usepackage{graphicx}
\makeatletter
\newsavebox\pandoc@box
\newcommand*\pandocbounded[1]{% scales image to fit in text height/width
  \sbox\pandoc@box{#1}%
  \Gscale@div\@tempa{\textheight}{\dimexpr\ht\pandoc@box+\dp\pandoc@box\relax}%
  \Gscale@div\@tempb{\linewidth}{\wd\pandoc@box}%
  \ifdim\@tempb\p@<\@tempa\p@\let\@tempa\@tempb\fi% select the smaller of both
  \ifdim\@tempa\p@<\p@\scalebox{\@tempa}{\usebox\pandoc@box}%
  \else\usebox{\pandoc@box}%
  \fi%
}
% Set default figure placement to htbp
\def\fps@figure{htbp}
\makeatother





\setlength{\emergencystretch}{3em} % prevent overfull lines

\providecommand{\tightlist}{%
  \setlength{\itemsep}{0pt}\setlength{\parskip}{0pt}}



 


\usepackage{graphicx}
\setkeys{Gin}{width=.95\linewidth,keepaspectratio}
\makeatletter
\@ifpackageloaded{caption}{}{\usepackage{caption}}
\AtBeginDocument{%
\ifdefined\contentsname
  \renewcommand*\contentsname{Table of contents}
\else
  \newcommand\contentsname{Table of contents}
\fi
\ifdefined\listfigurename
  \renewcommand*\listfigurename{List of Figures}
\else
  \newcommand\listfigurename{List of Figures}
\fi
\ifdefined\listtablename
  \renewcommand*\listtablename{List of Tables}
\else
  \newcommand\listtablename{List of Tables}
\fi
\ifdefined\figurename
  \renewcommand*\figurename{Figure}
\else
  \newcommand\figurename{Figure}
\fi
\ifdefined\tablename
  \renewcommand*\tablename{Table}
\else
  \newcommand\tablename{Table}
\fi
}
\@ifpackageloaded{float}{}{\usepackage{float}}
\floatstyle{ruled}
\@ifundefined{c@chapter}{\newfloat{codelisting}{h}{lop}}{\newfloat{codelisting}{h}{lop}[chapter]}
\floatname{codelisting}{Listing}
\newcommand*\listoflistings{\listof{codelisting}{List of Listings}}
\makeatother
\makeatletter
\makeatother
\makeatletter
\@ifpackageloaded{caption}{}{\usepackage{caption}}
\@ifpackageloaded{subcaption}{}{\usepackage{subcaption}}
\makeatother

\usepackage{bookmark}
\IfFileExists{xurl.sty}{\usepackage{xurl}}{} % add URL line breaks if available
\urlstyle{same}
\hypersetup{
  pdftitle={Measurement And Selection Biases},
  pdfauthor={Joseph A. Bulbulia},
  hidelinks,
  pdfcreator={LaTeX via pandoc}}


\title{Measurement And Selection Biases}
\author{Joseph A. Bulbulia}
\date{}

\begin{document}
\frame{\titlepage}


\begin{frame}
\section{Effect-Modification on Causal Directed Acyclic Graphs}

The primary function of a causal directed acyclic graph is to allow
investigators to apply Pearl's backdoor adjustment theorem to evaluate
whether causal effects may be identified from data, as shown in
Table\textasciitilde{}\ref{tbl-terminologygeneral}. We have noted that
modifying a causal effect within one or more strata of the target
population opens the possibility for biased average treatment effect
estimates when the distribution of these effect modifiers differs in the
analytic sample population \citep{bulbulia2024swigstime}.

We do not generally represent non-linearities in causal directed acyclic
graphs, which are tools for obtaining relationships of conditional and
unconditional independence from assumed structural relationships encoded
in a causal diagram that may lead to a non-causal treatment/outcome
association \citep{bulbulia2023}.

Table\textasciitilde{}\ref{tbl-effectmodification} presents our
convention for highlighting a relationship of effect modification in
settings where (1) we assume no confounding of treatment and outcome and
(2) there is effect modification such that the effect of \(A\) on \(Y\)
differs in at least one stratum of the target population.

\begin{table}[ht]
\centering
\terminologyeffectmodification
\caption{The five elementary structures of causality which all directed acyclic graphs are composed.}
\label{tbl-effectmodification}
\end{table}

To focus on effect modification, we do not draw a causal arrow from the
direct effect modifier \(F\) to the outcome \(Y\). This convention is
specific to this article (refer to \citealt{hernan2024WHATIF},
pp.~126--127, for a discussion of ``non-causal'\,' arrows).

\subsection{Part 1: How Measurement Error Bias Makes Your Causal Inferences \textbf{weird} (wrongly estimated inferences due to inappropriate restriction and distortion)}

Measurements record reality, but they are not always accurate. Whenever
variables are measured with error, our results can be misleading. Every
study must therefore consider how its measurements might mislead.

Causal graphs can deepen understanding because---as implied by the
concept of ``record'\,'---there are structural or causal properties that
give rise to measurement error. Measurement error can take various
forms, each with distinct implications for causal inference:

\begin{itemize}
    \item \textbf{Independent (undirected) / uncorrelated}: Errors in different variables do not influence each other.
    \item \textbf{Independent (undirected) and correlated}: Errors in different variables are related through a shared cause.
    \item \textbf{Dependent (directed) and uncorrelated}: Errors in one variable influence the measurement of another, but these influences are not related through a shared cause.
    \item \textbf{Dependent (directed) and correlated}: Errors in one variable influence the measurement of another, and these influences are related through a shared cause \citep{hernan2009, vanderweele2012a}.
\end{itemize}

The six causal diagrams presented in
Table\textasciitilde{}\ref{tbl-terminologymeasurementerror} illustrate
structural features of measurement error bias and clarify how these
structural features compromise causal inferences.

\begin{table}[ht]
\centering
\terminologymeasurementerror
\caption{Examples of measurement error bias.}
\label{tbl-terminologymeasurementerror}
\end{table}

Understanding these structural features will help explain why
measurement error bias cannot typically be evaluated with statistical
models, and will prepare us to link target-population restriction biases
to measurement error.

\subsubsection*{Example 1: Uncorrelated Non-Differential Errors under Sharp Null (No Treatment Effect)}

Table\textasciitilde{}\ref{tbl-terminologymeasurementerror}
\(\mathcal{G}_1\) illustrates uncorrelated non-differential measurement
error under the sharp null. In this setting, measurement error is not
expected to bias estimates.

\emph{Example:} A study on whether beliefs in big Gods affect social
complexity in ancient societies, where societies randomly omitted or
inaccurately recorded such beliefs and complexity, with errors
independent across variables. Under randomisation, uncorrelated
undirected errors will generally not bias estimates under the sharp
null, assuming all backdoor paths are closed. However, mismeasured
confounders can open backdoor paths \citep{robins2008estimation}.

\subsubsection*{Example 2: Uncorrelated Non-Differential Errors ``Off the Null'' (True Effect Present)}

Table\textasciitilde{}\ref{tbl-terminologymeasurementerror}
\(\mathcal{G}_2\) illustrates uncorrelated non-differential measurement
error when there is a true treatment effect, also called
\emph{information bias} \citep{lash2009applying}. This bias often
attenuates the effect toward the null---but not always
\citep{jurek2005proper, jurek2006exposure, jurek2008brief}.

\emph{Example:} Same as above, but a real effect exists. Measurement
error often underestimates it, but attenuation is not guaranteed.
Mismeasured confounders may still open backdoor paths.

\subsubsection*{Example 3: Correlated Non-Differential (Undirected) Measurement Errors}

Table\textasciitilde{}\ref{tbl-terminologymeasurementerror}
\(\mathcal{G}_3\) arises when the error terms of the treatment and
outcome share a common cause.

\emph{Example:} Societies with advanced record-keeping produce more
precise records of both big God beliefs and social complexity. This
common cause creates a spurious association even in the absence of a
true causal effect.

\subsubsection*{Example 4: Uncorrelated Differential Measurement Error — Exposure $\rightarrow$ Error in Outcome}

Table\textasciitilde{}\ref{tbl-terminologymeasurementerror}
\(\mathcal{G}_4\) occurs when the exposure influences how the outcome is
measured.

\emph{Example:} Big God beliefs lead to inflated historical records of
social complexity, introducing bias even without a true causal effect.

\subsubsection*{Example 5: Uncorrelated Differential Measurement Error — Outcome $\rightarrow$ Error in Exposure}

Table\textasciitilde{}\ref{tbl-terminologymeasurementerror}
\(\mathcal{G}_5\) occurs when the outcome influences the measurement of
the exposure.

\emph{Example:} If social complexity shapes historical narratives,
victors might record big God beliefs selectively to support political
legitimacy.

\subsubsection*{Example 6: Correlated Differential Measurement Error}

Table\textasciitilde{}\ref{tbl-terminologymeasurementerror}
\(\mathcal{G}_6\) occurs when the exposure influences already correlated
error terms.

\emph{Example:} Social complexity fosters elites who glorify both
political reach and big God beliefs, biasing both measures in a
correlated fashion.

\paragraph{Summary:}

In Part 1, we examined independent, correlated, dependent, and
correlated--dependent forms of measurement error bias. These structural
features clarify why such biases threaten causal inference and often
cannot be resolved with statistical adjustment alone
\citep{vanderweele2012a}.

\section{Part 2: Target Population Restriction Bias at the End of Study}\label{id-sec-2}

Suppose the analytic sample matches the target population at baseline.
Attrition (right-censoring) may bias causal effect estimates by:

\begin{enumerate}
    \item Opening biasing pathways (distortion), or
    \item Restricting the analytic sample so it is no longer representative (restriction).
\end{enumerate}

\begin{table}[ht]
\centering
\terminologycensoring
\caption{Five examples of right-censoring bias.}
\label{tbl-terminologycensoring}
\end{table}

\% Examples 1--5 follow here, unchanged in text but in LaTeX
paragraph/subsubsection form

\paragraph{Summary:}

Attrition can bias estimates whenever the distribution of effect
modifiers changes between baseline and study end. Methods such as
inverse probability weighting or multiple imputation can mitigate such
bias \citep{bulbulia2024PRACTICAL}, but only if their assumptions hold.
\end{frame}




\end{document}
